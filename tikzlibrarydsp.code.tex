% -------------------------------------------------------------------------
% This library for block diagrams and signal flow graphs was inspired by
% the library "signalflow" of Dr. Karlheinz Ochs, Ruhr-University of Bochum,
% Germany. Furthermore, some ideas were taken from the library "circuitikz"
% of Massimo A. Redaelli and from the PGF library itself.
%
% Copyright 2012 by Matthias Hotz
%
% This work is licensed under the Creative Commons Attribution 2.5 Generic
% License. To view a copy of this license, visit
%          http://creativecommons.org/licenses/by/2.5/
% or send a letter to Creative Commons, 444 Castro Street, Suite 900,
% Mountain View, California, 94041, USA.
%
%
% Modifications 2017 by Tycho Groche 
%
% -------------------------------------------------------------------------

\usetikzlibrary{arrows, 
	calc, 
	positioning, 
	decorations.markings, 
	shapes,
	decorations}

% -------------------------------------------------------------------------
% Parameters for the library

\newcommand{\dsplinewidth}{0.25mm}           % Line width for connections
\newcommand{\dspblocklinewidth}{0.3mm}       % Line width for blocks
\newcommand{\dspoperatordiameter}{4mm}       % Diameter for adder, multiplier, mixer
\newcommand{\dspoperatorlabelspacing}{2mm}   % Distance from symbol to label for adder, multiplier, mixer
\newcommand{\dspnoderadius}{1mm}             % Filled and empty node
\newcommand{\dspsquareblocksize}{8mm}        % Size for square blocks, e.g. for delay elements, decimator, expander
\newcommand{\dspfilterwidth}{14mm}           % Width of a filter block

% -------------------------------------------------------------------------
% Define new arrow heads

\pgfarrowsdeclare{dsparrow}{dsparrow}
{
	\arrowsize=0.25pt
	\advance\arrowsize by .5\pgflinewidth
	\pgfarrowsleftextend{-4\arrowsize}
	\pgfarrowsrightextend{4\arrowsize}
}
{
	\arrowsize=0.25pt
	\advance\arrowsize by .5\pgflinewidth
	\pgfsetdash{}{0pt} % Solid line (do not dash)
	\pgfsetmiterjoin	 % Fixed miter join of line
	\pgfsetbuttcap		 % Fixed butt cap of line
	\pgfpathmoveto{\pgfpoint{-4\arrowsize}{2.5\arrowsize}}
	\pgfpathlineto{\pgfpoint{4\arrowsize}{0pt}}
	\pgfpathlineto{\pgfpoint{-4\arrowsize}{-2.5\arrowsize}}
	\pgfpathclose
	\pgfusepathqfill
}

\pgfarrowsdeclare{dsparrowmid}{dsparrowmid}
{
	\arrowsize=0.25pt
	\advance\arrowsize by .5\pgflinewidth
	\pgfarrowsleftextend{-4\arrowsize}
	\pgfarrowsrightextend{4\arrowsize}
}
{
	\arrowsize=0.25pt
	\advance\arrowsize by .5\pgflinewidth
	\pgfsetdash{}{0pt}
	\pgfsetmiterjoin
	\pgfsetbuttcap
	\pgfpathmoveto{\pgfpoint{0}{2.5\arrowsize}}
	\pgfpathlineto{\pgfpoint{8\arrowsize}{0pt}}
	\pgfpathlineto{\pgfpoint{0}{-2.5\arrowsize}}
	\pgfpathclose
	\pgfusepathqfill
}

% -------------------------------------------------------------------------
% Define new node shapes

\makeatletter


\pgfkeys{/tikz/dsp/label/.initial=above}

% Generic shape generator for operators, i.e. nodes with a circular
% shape with an additional customizable drawing and a text label
\long\def\dspdeclareoperator#1#2{
	\pgfdeclareshape{#1}
	{
		% Saved anchors, macros and dimensions
		\savedanchor\centerpoint{\pgfpointorigin}
		\savedmacro\label{\def\label{\pgfkeysvalueof{/tikz/dsp/label}}}
	  \saveddimen\radius
	  {
		  \pgfmathsetlength\pgf@xa{\pgfshapeminwidth}
		  \pgfmathsetlength\pgf@ya{\pgfshapeminheight}
	    \ifdim\pgf@xa>\pgf@ya
	      \pgf@x=.5\pgf@xa
	    \else
	      \pgf@x=.5\pgf@ya
	    \fi
	  }
	  
	  \foreach \x in {center, mid, mid west, mid east, base, base west, base east,
	  	north, north east, east, south east, south, south west, west, north west}{
	  	\inheritanchor[from={circle}]{\x}
	  }
	  \inheritanchorborder[from={circle}]
	  	  
	  % Draw circle and embed additional code
	  \backgroundpath
	  {
	    % Draw circle
	    \pgfpathcircle{\centerpoint}{\radius}
	    
	    % Embed additional code
	    % (Note that this code must call e.g. \pgfusepathqstroke
	    #2
	  }
	
		% Define anchor parametrized by the PGF key /tikz/dsp/label
	  \anchor{text}
	  {
			\centerpoint
			%
	    \def\templabelabove{above}
	    \def\templabelbelow{below}
	    \def\templabelleft{left}
	    \def\templabelright{right}
	    \pgfutil@tempdima=\dspoperatorlabelspacing
	    %
	    \ifx\label\templabelabove
				\advance\pgf@x by -0.5\wd\pgfnodeparttextbox
				\advance\pgf@y by \radius
				\advance\pgf@y by \pgfutil@tempdima
	    \fi
	    %
	    \ifx\label\templabelbelow
				\advance\pgf@x by -0.5\wd\pgfnodeparttextbox
				\advance\pgf@y by -\radius
				\advance\pgf@y by -\pgfutil@tempdima
				\advance\pgf@y by -\ht\pgfnodeparttextbox
	    \fi
	    %
	    \ifx\label\templabelleft
				\advance\pgf@x by -\radius
				\advance\pgf@x by -\pgfutil@tempdima
				\advance\pgf@x by -\wd\pgfnodeparttextbox
				\advance\pgf@y by -0.5\ht\pgfnodeparttextbox
				\advance\pgf@y by +0.5\dp\pgfnodeparttextbox
	    \fi
	    %
	    \ifx\label\templabelright
				\advance\pgf@x by \radius
				\advance\pgf@x by \pgfutil@tempdima
				\advance\pgf@y by -0.5\ht\pgfnodeparttextbox
				\advance\pgf@y by +0.5\dp\pgfnodeparttextbox
	    \fi
	  }
	}
}

\dspdeclareoperator{dspshapecircle}{\pgfusepathqstroke}

\dspdeclareoperator{dspshapecirclefull}{\pgfusepathqfillstroke}

\dspdeclareoperator{dspshapeadder}{
	% Coordinate offset for the plus
	\pgfutil@tempdima=\radius
	\pgfutil@tempdima=0.55\pgfutil@tempdima
	
	% Draw plus
	\pgfmoveto{\pgfpointadd{\centerpoint}{\pgfpoint{0pt}{-\pgfutil@tempdima}}}
	\pgflineto{\pgfpointadd{\centerpoint}{\pgfpoint{0pt}{ \pgfutil@tempdima}}}
	
	\pgfmoveto{\pgfpointadd{\centerpoint}{\pgfpoint{-\pgfutil@tempdima}{0pt}}}
	\pgflineto{\pgfpointadd{\centerpoint}{\pgfpoint{ \pgfutil@tempdima}{0pt}}}
	
	\pgfusepathqstroke
}

\dspdeclareoperator{dspshapemixer}{
	% Coordinate offset for the cross
	\pgfutil@tempdima=\radius
	\pgfutil@tempdima=0.707106781\pgfutil@tempdima
	
	% Draw cross
	\pgfmoveto{\pgfpointadd{\centerpoint}{\pgfpoint{-\pgfutil@tempdima}{-\pgfutil@tempdima}}}
	\pgflineto{\pgfpointadd{\centerpoint}{\pgfpoint{ \pgfutil@tempdima}{ \pgfutil@tempdima}}}
	
	\pgfmoveto{\pgfpointadd{\centerpoint}{\pgfpoint{-\pgfutil@tempdima}{ \pgfutil@tempdima}}}
	\pgflineto{\pgfpointadd{\centerpoint}{\pgfpoint{ \pgfutil@tempdima}{-\pgfutil@tempdima}}}
	
	\pgfusepathqstroke
}

\pgfdeclareshape{dspshapesaturation}{
	
	\inheritsavedanchors[from={rectangle}]
	\inheritbackgroundpath[from={rectangle}]
	\inheritanchorborder[from={rectangle}]
	\foreach \x in {center, mid, mid west, mid east, base, base west, base east,
		north, north east, east, south east, south, south west, west, north west}{
		\inheritanchor[from={rectangle}]{\x}
	}

	\foregroundpath{
		\pgfpointdiff{\northeast}{\southwest}
		\pgf@xa=\pgf@x \pgf@ya=\pgf@y
		\northeast
		\pgfpathmoveto{\pgfpoint{0}{0.33\pgf@ya}}
		\pgfpathlineto{\pgfpoint{0}{-0.33\pgf@ya}}
		\pgfpathmoveto{\pgfpoint{0.33\pgf@xa}{0}}
		\pgfpathlineto{\pgfpoint{-0.33\pgf@xa}{0}}
		\pgfpathmoveto{\pgfpointadd{\southwest}{\pgfpoint{-0.33\pgf@xa}{-0.6\pgf@ya}}}
		\pgfpathlineto{\pgfpointadd{\southwest}{\pgfpoint{-0.5\pgf@xa}{-0.6\pgf@ya}}}
		\pgfpathlineto{\pgfpointadd{\northeast}{\pgfpoint{-0.5\pgf@xa}{-0.6\pgf@ya}}}
		\pgfpathlineto{\pgfpointadd{\northeast}{\pgfpoint{-0.33\pgf@xa}{-0.6\pgf@ya}}}
	}
}

\pgfdeclareshape{dspshapesinewave}{
	
	\inheritsavedanchors[from={rectangle}]
	\inheritbackgroundpath[from={rectangle}]
	\inheritanchorborder[from={rectangle}]
	\foreach \x in {center, mid, mid west, mid east, base, base west, base east,
		north, north east, east, south east, south, south west, west, north west}{
		\inheritanchor[from={rectangle}]{\x}
	}
	
	\foregroundpath{
		\pgfpointdiff{\northeast}{\southwest}
		\pgf@xa=\pgf@x \pgf@ya=\pgf@y
		\northeast

		\pgfpathmoveto{\pgfpoint{0}{0.33\pgf@ya}}
		\pgfpathlineto{\pgfpoint{0}{-0.33\pgf@ya}}
		\pgfpathmoveto{\pgfpoint{0.33\pgf@xa}{0}}
		\pgfpathlineto{\pgfpoint{-0.33\pgf@xa}{0}}

		\pgfpathmoveto{\pgfpoint{0.3\pgf@xa}{0}}
		\pgfpathsine{\pgfpoint{-0.15\pgf@xa}{0.25\pgf@ya}}
		\pgfpathcosine{\pgfpoint{-0.15\pgf@xa}{-0.25\pgf@ya}}
		\pgfpathsine{\pgfpoint{-0.15\pgf@xa}{-0.25\pgf@ya}}
		\pgfpathcosine{\pgfpoint{-0.15\pgf@xa}{0.25\pgf@ya}}
	}
}

\pgfdeclareshape{dspshapeadaptive}{
	
	\inheritsavedanchors[from={rectangle}]
	\inheritbackgroundpath[from={rectangle}]
	\inheritanchorborder[from={rectangle}]
	\foreach \x in {center, mid, mid west, mid east, base, base west, base east,
		north, north east, east, south east, south, south west, west, north west}{
		\inheritanchor[from={rectangle}]{\x}
	}
	
	\anchor{arrowstart}
	{	
		\pgfpointdiff{\northeast}{\southwest}
		\pgf@xa=\pgf@x \pgf@ya=\pgf@y
		\southwest
		\advance\pgf@x by -0.05\pgf@xa
		\advance\pgf@y by 0.3\pgf@ya
	}

	\anchor{arrowend}
	{
		\pgfpointdiff{\northeast}{\southwest}
		\pgf@xa=\pgf@x \pgf@ya=\pgf@y
		\northeast
		\advance\pgf@x by 0.05\pgf@xa
		\advance\pgf@y by -0.3\pgf@ya
	}
	
	\behindbackgroundpath
	{
		\pgfpointdiff{\northeast}{\southwest}
		\pgf@xa=\pgf@x \pgf@ya=\pgf@y
		
		\southwest
		\advance\pgf@x by -0.05\pgf@xa
		\advance\pgf@y by 0.3\pgf@ya
		\pgfpathmoveto{\pgfpoint{\pgf@x}{\pgf@y}}
		
		\northeast
		\advance\pgf@x by 0.05\pgf@xa
		\advance\pgf@y by -0.3\pgf@ya
		
		\pgfpathlineto{\pgfpoint{\pgf@x}{\pgf@y}}
		\pgfsetarrowsend{latex}
		\pgfsetlinewidth{\dsplinewidth}
		\pgfsetroundcap
		\pgfusepath{stroke}
	}

	\beforebackgroundpath
	{
    	\pgfpathrectanglecorners
		{\pgfpointadd{\southwest}{\pgfpoint{\dspblocklinewidth}{\dspblocklinewidth}}}
		{\pgfpointadd{\northeast}{\pgfpointscale{-1}{\pgfpoint{\dspblocklinewidth}{\dspblocklinewidth}}}}
		
		\color{white}
		\pgfusepath{fill}
	}
		
	
}


\pgfdeclareshape{dspshapemic}{

	\inheritsavedanchors[from={circle}]
	\inheritbackgroundpath[from={circle}]
	\inheritanchorborder[from={circle}]
	\foreach \x in {center, mid, mid west, mid east, base, base west, base east,
		north, north east, east, south east, south, south west, west, north west}	{
		\inheritanchor[from={circle}]{\x}
	}


	\beforebackgroundpath{	
	\pgfmathsetlength\pgf@xa{\radius}
	
	\pgfpathmoveto{\pgfpoint{\pgf@xa}{\pgf@xa}}
	\pgfpathlineto{\pgfpoint{\pgf@xa}{-\pgf@xa}}
	
	\pgfusepathqstroke
	}
}


\pgfdeclareshape{dspshapespeaker}{
	
	\inheritsavedanchors[from={rectangle}]
	\inheritanchorborder[from={rectangle}]
	\foreach \x in {center, mid, mid west, mid east, base, base west, base east,
		north, north east, east, south east, south, south west, west, north west}{
		\inheritanchor[from={rectangle}]{\x}
	}

	\anchor{speaker}
	{
		\pgfpointdiff{\northeast}{\southwest}
		\pgf@xa=\pgf@x \pgf@ya=\pgf@y	
		\northeast
		\advance\pgf@x by -\pgf@xa
		\advance\pgf@y by 0.5\pgf@ya
	}

	\backgroundpath{
		\pgfpointdiff{\northeast}{\southwest}
		\pgf@xc=\pgf@x \pgf@yc=\pgf@y
		
		\pgfpathmoveto{\northeast}	
		\pgfpathlineto{\pgfpointadd{\northeast}{\pgfpoint{-\pgf@xc}{-0.5\pgf@yc}}}
		\pgfpathlineto{\pgfpointadd{\northeast}{\pgfpoint{-\pgf@xc}{1.5\pgf@yc}}}
		\pgfpathlineto{\pgfpointadd{\northeast}{\pgfpoint{0}{1\pgf@yc}}}
		
		\pgfpathrectanglecorners{\southwest}{\northeast}
		
	}

}

% -------------------------------------------------------------------------
% Decoration for corner-aligned dashed border
%--------------------------------------------------------------------------
%https://tex.stackexchange.com/questions/280187/create-corner-aligned-dashed-rectangles-using-tikz

\def\pgf@dec@dashon{5pt}
\def\pgf@dec@dashoff{5pt}
\pgfkeys{/pgf/decoration/.cd,
	/pgf/decoration/dash on/.store in=\pgf@dec@dashon,
	/pgf/decoration/dash off/.store in=\pgf@dec@dashoff
}
\pgfdeclaredecoration{aligned dash}{start}{
	\state{start}[width=0pt, next state=pre-corner,persistent precomputation={
		\pgfextract@process\pgffirstpoint{\pgfpointdecoratedinputsegmentfirst}%
		\pgfextract@process\pgfsecondpoint{\pgfpointdecoratedinputsegmentlast}%
		\pgfmathsetlengthmacro\pgf@dec@dashon{\pgf@dec@dashon}%
		\pgfmathsetlengthmacro\pgf@dec@dashoff{\pgf@dec@dashoff}%
		\pgfmathsetlengthmacro\pgf@dec@halfdash{\pgf@dec@dashon/2}%
	}]{}
	\state{pre-corner}[width=\pgfdecoratedinputsegmentlength, next state=post-corner, persistent precomputation={
		%
		\pgfmathparse{int(ceil((\pgfdecoratedinputsegmentlength-\pgf@dec@dashon-\pgf@dec@dashoff)/(\pgf@dec@dashon+\pgf@dec@dashoff)))}%
		\let\pgf@n=\pgfmathresult
		\pgfmathsetlengthmacro\pgf@b%
		{\pgfdecoratedinputsegmentlength/(\pgf@n+1)-\pgf@dec@dashon}%
		\ifdim\pgf@b<\pgf@dec@dashoff\relax%
		\pgfmathparse{int(\pgf@n-1)}\let\pgf@n=\pgfmathresult%
		\pgfmathsetlengthmacro\pgf@b%
		{\pgfdecoratedinputsegmentlength/(\pgf@n+1)-\pgf@dec@dashon}%
		\fi%
		\pgfmathsetlengthmacro\pgf@b{\pgf@b+\pgf@dec@dashon}%
	}]{%
		\pgfmathloop
		\ifnum\pgfmathcounter>\pgf@n%
		\else%
		\pgfpathmoveto{\pgfpoint{\pgf@b*\pgfmathcounter-\pgf@dec@halfdash}{0pt}}%
		\pgfpathlineto{\pgfpoint{\pgf@b*\pgfmathcounter+\pgf@dec@halfdash}{0pt}}%
		\repeatpgfmathloop%
		\pgfpathmoveto%
		{\pgfpoint{\pgfdecoratedinputsegmentlength-\pgf@dec@halfdash}{0pt}}%
		\pgfpathlineto%
		{\pgfpointdecoratedinputsegmentlast}
	}
	\state{post-corner}[width=0pt, next state=pre-corner]{
		\pgfpathlineto{\pgfpoint{\pgf@dec@halfdash}{0pt}}%
	}
	\state{final}{
		\pgftransformreset%
		\pgfpathlineto{\pgfpointlineatdistance{\pgf@dec@halfdash}{\pgffirstpoint}{\pgfsecondpoint}}%
	}
}


\makeatother

% -------------------------------------------------------------------------
% Define border styles
%--------------------------------------------------------------------------

\tikzset{aligned dash/.style=
	{
		decoration=
		{
			aligned dash, 
			#1
		}, 
		decorate
	}
}


% -------------------------------------------------------------------------
% Define node styles
%--------------------------------------------------------------------------

\tikzset{dspadder/.style=
	{
		shape=dspshapeadder,
		line cap=rect,
		line join=rect,
		line width=\dspblocklinewidth,
		minimum size=\dspoperatordiameter
	}
}


\tikzset{dspmultiplier/.style=
	{
		shape=dspshapecircle,
		line cap=rect
		,line join=rect,
		line width=\dspblocklinewidth,
		minimum size=\dspoperatordiameter
	}
}


\tikzset{dspmixer/.style=
	{
		shape=dspshapemixer,
		line cap=rect,
		line join=rect,
		line width=\dspblocklinewidth,
		minimum size=\dspoperatordiameter
	}
}


\tikzset{dspnodeopen/.style=
	{
		shape=dspshapecircle,
		line width=\dsplinewidth,
		minimum size=\dspnoderadius
	}
}


\tikzset{dspnodefull/.style=
	{
		shape=dspshapecirclefull,
		line width=\dsplinewidth,
		fill,
		minimum size=\dspnoderadius
	}
}

% The fixed specification of text height and text depth is the somewhat
% unaesthetic workaround to align the text in different node at the same
% baseline. See the PGF/TikZ manual, ch. 5.1.
\tikzset{dspsquare/.style=
	{
		shape=rectangle,
		draw,
		align=center,
		text depth=0.3em,
		text height=1em,
		inner sep=0pt, 
		line cap=round,
		line join=round,
		line width=\dspblocklinewidth,
		minimum size=\dspsquareblocksize
	}
}


\tikzset{dspfilter/.style=
	{
		shape=rectangle,
		draw,
		align=center,
		text depth=0.3em,
		text height=1em,
		inner sep=0pt,
		line cap=round,
		line join=round,
		line width=\dsplinewidth,
		minimum height=\dspsquareblocksize,
		minimum width=\dspfilterwidth
	}
}


\tikzset{dspgain/.style=
	{
		draw,
		isosceles triangle, 
		isosceles triangle apex angle=60,
		minimum width = \dspsquareblocksize,  
		inner sep = 0pt
	}
}
	

\tikzset{dspsaturate/.style=
	{
		draw,
		shape=dspshapesaturation,
		line width=\dspblocklinewidth, 
		minimum size=\dspsquareblocksize
	}
}


\tikzset{dspsinewave/.style=
	{
		draw,
		shape=dspshapesinewave, 
		line width=\dspblocklinewidth, 
		minimum size=\dspsquareblocksize
	}
}


\tikzset{dspadaptivefilter/.style=
	{
		draw,
		shape=dspshapeadaptive,
		line width=\dspblocklinewidth,
		minimum height = \dspsquareblocksize, 
		minimum width=\dspfilterwidth
	}
}


\tikzset{dspsystem/.style=
	{
		draw,
		line width = 0.5*\dspblocklinewidth,
		inner sep = 10pt,
		aligned dash =
		{
			%
		}
	}
}


\tikzset{dspmic/.style=
	{
		draw,
		shape = dspshapemic,
		line width = \dspblocklinewidth,
		minimum size = \dspoperatordiameter
	}
}


\tikzset{dspspeaker/.style=
	{
		draw,
		shape = dspshapespeaker,
		line width = \dspblocklinewidth,
	}
}
% -------------------------------------------------------------------------
% Define "signal flow" lines
%--------------------------------------------------------------------------

\tikzset{dspline/.style=
	{
		line width=\dsplinewidth
	},
	line cap=round,
	line join=round
}


\tikzset{dspconn/.style=
	{
		->,
		>=dsparrow,
		line width=\dsplinewidth
	},
	line cap=round,
	line join=round
}


\tikzset{dspflow/.style=
	{
		line width=\dsplinewidth,
		line cap=round,
		line join=round,
		decoration=
		{
			markings,
			mark=at position 0.5 with 
			{
				\arrow{dsparrowmid}
			}
		},
		postaction={decorate}
	}
}

% -------------------------------------------------------------------------
% Define various utility macros

\newcommand{\downsamplertext}[1]{\raisebox{0.1em}{$\big\downarrow$}#1}
\newcommand{\upsamplertext}[1]{\raisebox{0.1em}{$\big\uparrow$}#1}