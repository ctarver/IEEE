% This template has been tested with IEEEtran of 2015.

% !TeX spellcheck = en-US
% !TeX encoding = utf8
% !TeX program = pdflatex
% !BIB program = bibtex
% -*- coding:utf-8 mod:LaTeX -*-

%cmap has to be loaded before any font package (such as newtxmath)
\RequirePackage{cmap}

% DO NOT DOWNLOAD IEEEtran.cls - Use the one of your LaTeX distribution
\documentclass[conference]{IEEEtran}[2015/08/26]

% use nicer font for code
\usepackage[zerostyle=b,scaled=.75]{newtxtt}

\usepackage{tikz}
\usetikzlibrary{shapes.geometric, arrows, dsp, chains}
\def\layersep{1.5cm}

\usepackage{pgfplots}
\pgfplotsset{compat=newest}
\usepgfplotslibrary{groupplots}
\usepgfplotslibrary{dateplot}

% for demonstration purposes
\usepackage{mwe}

\usepackage[T1]{fontenc}
\usepackage[utf8]{inputenc} %support umlauts in the input

\usepackage{amsmath,amssymb,amsfonts}
\usepackage{algorithm}
\usepackage{algorithmicx}
\usepackage{algpseudocode}
\renewcommand{\algorithmiccomment}[1]{\bgroup\hfill//~#1\egroup}
\usepackage{graphicx}

%Set English as language and allow to write hyphenated"=words
\usepackage[ngerman,main=english]{babel}
%Hint by http://tex.stackexchange.com/a/321066/9075 -> enable "= as dashes
\addto\extrasenglish{\languageshorthands{ngerman}\useshorthands{"}}

% backticks (`) are rendered as such in verbatim environment. See https://tex.stackexchange.com/a/341057/9075 for details.
\usepackage{upquote}

%extended enumerate, such as \begin{compactenum}
\usepackage{paralist}

%for easy quotations: \enquote{text}
\usepackage{csquotes}

%enable margin kerning
\RequirePackage{iftex}
\ifPDFTeX
  \RequirePackage[%
    final,%
    expansion=alltext,%
    protrusion=alltext-nott]{microtype}%
\else
  \RequirePackage[%
    final,%
    protrusion=alltext-nott]{microtype}%
\fi%
% \texttt{test -- test} keeps the "--" as "--" (and does not convert it to an en dash)
\DisableLigatures{encoding = T1, family = tt* }

%tweak \url{...}
\usepackage{url}
%\urlstyle{same}
%improve wrapping of URLs - hint by http://tex.stackexchange.com/a/10419/9075
\makeatletter
\g@addto@macro{\UrlBreaks}{\UrlOrds}
\makeatother
%nicer // - solution by http://tex.stackexchange.com/a/98470/9075
%DO NOT ACTIVATE -> prevents line breaks
%\makeatletter
%\def\Url@twoslashes{\mathchar`\/\@ifnextchar/{\kern-.2em}{}}
%\g@addto@macro\UrlSpecials{\do\/{\Url@twoslashes}}
%\makeatother

% Diagonal lines in a table - http://tex.stackexchange.com/questions/17745/diagonal-lines-in-table-cell
% Slashbox is not available in texlive (due to licensing) and also gives bad results. This, we use diagbox
%\usepackage{diagbox}

%https://people.inf.ethz.ch/markusp/teaching/guides/guide-tables.pdf
\usepackage{booktabs}
\renewcommand{\arraystretch}{1.2} % More space between rows

% Required for package pdfcomment later
\usepackage{xcolor}

\usepackage[nolist]{acronym}
\makeatletter
\renewcommand{\ALG@name}{Kernel}
\makeatother

% For listings
\usepackage{listings}
\lstset{%
  basicstyle=\ttfamily,%
  columns=fixed,%
  basewidth=.5em,%
  xleftmargin=0.5cm,%
  captionpos=b}%

% Enable nice comments
\usepackage{pdfcomment}
%
\newcommand{\commentontext}[2]{\colorbox{yellow!60}{#1}\pdfcomment[color={0.234 0.867 0.211},hoffset=-6pt,voffset=10pt,opacity=0.5]{#2}}
\newcommand{\commentatside}[1]{\pdfcomment[color={0.045 0.278 0.643},icon=Note]{#1}}
%
% Compatibility with packages todo, easy-todo, todonotes
\newcommand{\todo}[1]{\commentatside{#1}}
% Compatiblity with package fixmetodonotes
\newcommand{\TODO}[1]{\commentatside{#1}}

% Bibliopgraphy enhancements
%  - enable \cite[prenote][]{ref}
%  - enable \cite{ref1,ref2}
% Alternative: \usepackage{cite}, which enables \cite{ref1, ref2} only (otherwise: Error message: "White space in argument")
%
% Doc: http://texdoc.net/natbib
\ifCLASSOPTIONcompsoc
  % IEEE Computer Society needs nocompress option at cite.sty
  % natbib includes the same functionality
  \usepackage[%
    square,        % for square brackets
    comma,         % use commas as separators
    numbers,       % for numerical citations;
    sort           % orders multiple citations into the sequence in which they appear in the list of references;
    %sort&compress % as sort but in addition multiple numerical citations
                   % are compressed if possible (as 3-6, 15);
  ]{natbib}
\else
  % normal IEEE
  \usepackage[%
    square,        % for square brackets
    comma,         % use commas as separators
    numbers,       % for numerical citations;
    %sort           % orders multiple citations into the sequence in which they appear in the list of references;
    sort&compress % as sort but in addition multiple numerical citations
                   % are compressed if possible (as 3-6, 15);
  ]{natbib}
\fi
% Same fontsize as without natbib
\renewcommand{\bibfont}{\normalfont\footnotesize}
% Enable hyperlinked author names in the case of \citet
% Source: https://tex.stackexchange.com/a/76075/9075
\usepackage{etoolbox}
\makeatletter
\patchcmd{\NAT@test}{\else \NAT@nm}{\else \NAT@hyper@{\NAT@nm}}{}{}
\makeatother

% Enable that parameters of \cref{}, \ref{}, \cite{}, ... are linked so that a reader can click on the number an jump to the target in the document
\usepackage{hyperref}
% Enable hyperref without colors and without bookmarks
\hypersetup{hidelinks,
  colorlinks=true,
  allcolors=black,
  pdfstartview=Fit,
  breaklinks=true}
%
% Enable correct jumping to figures when referencing
\usepackage[all]{hypcap}

%enable \cref{...} and \Cref{...} instead of \ref: Type of reference included in the link
\usepackage[capitalise,nameinlink]{cleveref}
\crefname{lstlisting}{\lstlistingname}{\lstlistingname}
\Crefname{lstlisting}{Listing}{Listings}

%Following definitions are outside of IfPackageLoaded; inside, they are not visible
%
%Intermediate solution for hyperlinked refs. See https://tex.stackexchange.com/q/132420/9075 for more information.
\newcommand{\Vlabel}[1]{\label[line]{#1}\hypertarget{#1}{}}
\newcommand{\lref}[1]{\hyperlink{#1}{\FancyVerbLineautorefname~\ref*{#1}}}

\newenvironment{listing}[1][htbp!]{\begin{figure}[#1]}{\end{figure}}
\newcounter{listing}

\usepackage{xspace}
%\newcommand{\eg}{e.\,g.\xspace}
%\newcommand{\ie}{i.\,e.\xspace}
\newcommand{\eg}{e.\,g.,\ }
\newcommand{\ie}{i.\,e.,\ }

%introduce \powerset - hint by http://matheplanet.com/matheplanet/nuke/html/viewtopic.php?topic=136492&post_id=997377
\DeclareFontFamily{U}{MnSymbolC}{}
\DeclareSymbolFont{MnSyC}{U}{MnSymbolC}{m}{n}
\DeclareFontShape{U}{MnSymbolC}{m}{n}{
  <-6>    MnSymbolC5
  <6-7>   MnSymbolC6
  <7-8>   MnSymbolC7
  <8-9>   MnSymbolC8
  <9-10>  MnSymbolC9
  <10-12> MnSymbolC10
  <12->   MnSymbolC12%
}{}
\DeclareMathSymbol{\powerset}{\mathord}{MnSyC}{180}

% *** SUBFIGURE PACKAGES ***
\ifCLASSOPTIONcompsoc
  \usepackage[caption=false,font=footnotesize,labelfont=sf,textfont=sf]{subfig}
\else
  \usepackage[caption=false,font=footnotesize]{subfig}
\fi

\usepackage{stfloats}

% correct bad hyphenation here
\hyphenation{op-tical net-works semi-conduc-tor}


\newcommand{\norm}[1]{\left\lVert#1\right\rVert}


% Make some colors that we want to use throughout paper
\definecolor{nn}{RGB}{35,139,69}
\definecolor{m2}{RGB}{254,178,76} 
\definecolor{m1}{RGB}{254,217,118} 
\definecolor{nn2}{RGB}{34,94,168} 
\definecolor{m4}{RGB}{252,197,192}
\definecolor{m4pq}{RGB}{228,26,28} 

\begin{document}
%\IEEEoverridecommandlockouts

\title{Quick start for LaTeXing with IEEEtran.cls for\\ IEEE Conferences}

\author{\IEEEauthorblockN{Chance Tarver\IEEEauthorrefmark{1},
		Christoph\IEEEauthorrefmark{2},
		and
		Joseph R. Cavallaro\IEEEauthorrefmark{2}}
	\IEEEauthorblockA{\IEEEauthorrefmark{1}\IEEEauthorrefmark{2}Department of Electrical and Computer Engineering,
		Rice University, 
		Houston, TX, USA\\}
	\IEEEauthorblockA{\IEEEauthorrefmark{1}Standard and Mobility Innovation Lab, 
		Samsung Research America,
		Plano, TX, USA}		
	
}

% use for special paper notices
%\IEEEspecialpapernotice{(Invited Paper)}

% make the title area
\maketitle

% In case you want to add a copyright statement.
%
% Source: https://tex.stackexchange.com/a/200330/9075
%
% All possible solutions:
%  - https://tex.stackexchange.com/a/325013/9075
%  - https://tex.stackexchange.com/a/279134/9075
%  - https://tex.stackexchange.com/q/279789/9075 (TikZ)
%  - https://tex.stackexchange.com/a/200330/9075 - for non-compsocc papers
\iffalse
    \makeatletter
    \def\ps@IEEEtitlepagestyle{%
      \def\@oddfoot{\mycopyrightnotice}%
      \def\@evenfoot{}%
    }
    \makeatother
    \def\mycopyrightnotice{%
      \begin{minipage}{\textwidth}
        \footnotesize
        1551-3203 \copyright 2015 IEEE.
        Personal use is permitted, but republication/redistribution requires IEEE permission.
        \\
        See \url{https://www.ieee.org/publications_standards/publications/rights/index.html} for more information.
      \end{minipage}
      \gdef\mycopyrightnotice{}% just in case
    }
\fi

\begin{abstract}
  \blindtext
\end{abstract}

% For peer review papers, you can put extra information on the cover
% page as needed:
% \ifCLASSOPTIONpeerreview
% \begin{center} \bfseries EDICS Category: 3-BBND \end{center}
% \fi
%
% For peerreview papers, this IEEEtran command inserts a page break and
% creates the second title. It will be ignored for other modes.
\IEEEpeerreviewmaketitle

\begin{acronym}
	% Initialisms and Acronyms
	\acro{GPU}{graphics processing unit}
	\acro{SDR}{software-defined radio}
	\acro{LDPC}{low-density parity-check}
	\acro{NR}{New Radio}
	\acro{OAI}{Open Air Interface}
	\acro{RAN}{radio access network}
	\acro{C-RAN}{cloud radio access network}
	\acro{MU}{multi-user}
	\acro{MIMO}{multiple-input multiple-output}
	\acro{CUDA}{compute unified device architecture}
	\acro{DVB}{Digital Video Broadcasting}
	\acro{CN}{check node}
	\acro{VN}{variable node}
	\acro{DL-SCH}{downlink shared channel}
	\acro{UL-SCH}{uplink shared channel}
	\acro{PCH}{paging channel}
	\acro{gNB}{next-generation NodeB}
	\acro{BCH}{broadcast channel}
	\acro{BG}{base graph}
	\acro{MCS}{modulation and coding scheme}
	\acro{QC}{quasi-cyclic}
	\acro{CRC}{cyclic redundancy check }
	\acro{BER}{bit error rate}
	\acro{FER}{frame error rate}
	\acro{eMBB}{enhanced mobile broadband}
	\acro{URLLC}{ultra-reliable low-latency communications}
	\acro{mMTC}{massive machine-type communications}
	\acro{SNR}{signal-to-noise ratio}
	\acro{LLR}{log-likelihood ratio}
	\acro{SM}{streaming multiprocessor}
	\acro{APP}{a posteriori probabilities}
	\acro{SPA}{sum-product algorithm}
	\acro{MSA}{min-sum algorithm}
	\acro{COTS}{commercially available off-the-shelf}
	\acro{UE}{user equipment}
	\acro{CoMP}{coordinated multipoint}
	\acro{ECC}{error correction codes}
	\acro{SIMT}{single-instruction multiple-thread}
	\acro{RAT}{radio access technology}
	\acroplural{RAT}[RATs]{radio access technologies}
	\acro{vRAN}{virtualized radio access networks}
	\acro{E$_b$/N$_0$}{energy per bit to noise power spectral density}
	\acro{CW}{codeword}
	\acro{MCW}{macro-codeword}
	\acro{PA}{power amplifier}
	\acro{ACLR}{adjacent channel leakage ratio}
	\acro{EVM}{error vector magnitde}
	\acro{MIMO}{multiple-input multiple-output}
	\acro{DPD}{digital predistortion}
	\acro{FPGA}{field programmable gate array}
	\acro{MP}{memory polynomial}
	\acro{DAC}{digital-to-analog converter}
	\acro{NIC}{network interface controller}
\end{acronym}

\section{Introduction}
\label{sec:intro}

\lipsum[1-3]

The remainder of the paper starts with a presentation of related work (\cref{sec:relatedwork}).
It is followed by a presentation of hints on \LaTeX{} (\cref{sec:hints}).
Based on that, we present some Lorem Ipsum (\cref{sec:loremipsum}).
Finally, a conclusion is drawn and outlook on future work is made (\cref{sec:outlook}).

\begin{align}
\label{eq:mp}
\hat{x}^{(i)}(n) = \sum_{p = 1}^{P} \sum_{m = 0}^{M} \beta^{(i)}_{p,m} x^{(i)}(n-m) \left| x^{(i)}(n-m) \right|^{p-1},
\end{align}

There is an equation in \eqref{eq:mp}.


\section{Related Work}
\label{sec:relatedwork}

Winery~\cite{2019_mikkogpu} is a graphical \commentontext{modeling}{modeling with one \enquote{l}, because of AE} tool.
The whole idea of TOSCA is explained by \citet{memory_polynomial}.

\begin{figure}
	% This file was created by matlab2tikz.
%
%The latest updates can be retrieved from
%  http://www.mathworks.com/matlabcentral/fileexchange/22022-matlab2tikz-matlab2tikz
%where you can also make suggestions and rate matlab2tikz.
%
\definecolor{mycolor1}{rgb}{0.00000,0.44700,0.74100}%
%
\begin{tikzpicture}

\begin{axis}[%
reverse legend,
width=6.8cm,height=4.8cm,
scale only axis,
xmin=0,
xmax=125,
xmajorgrids,
ymajorgrids,
axis background/.style={fill=white},
xlabel={Number of Real Multiplications},
ylabel={EVM (\%)},
legend style={legend cell align=left, align=left, draw=white!15!black, font=\small},
every axis plot/.append style={thick}
]

\addplot [color=m4, mark=*, mark options={solid,m4}]
table[row sep=crcr]{%
	12	3.0103\\
	28	2.7322\\
	45	2.3089\\
	63	2.0085\\
	82	1.9722\\
	102	1.95\\
	123	1.9552\\
};
\addlegendentry{$M=4$}


\addplot [color=m2, mark=*, mark options={solid,m2}]
table[row sep=crcr]{%
	6	3.0388192857864\\
	16	2.73656091785756\\
	27	2.3154671075041\\
	39	2.00375004718572\\
	52	1.95769885368691\\
	66	1.95947582666292\\
	81	1.95781082858679\\
};
\addlegendentry{$M=2$}

\addplot [color=m1, mark=*, mark options={solid, m1}]
table[row sep=crcr]{%
	3	3.0375\\
	10	2.751\\
	18	2.3392\\
	27	2.0256\\
	37	1.9577\\
	48	1.9814\\
	60	1.9837\\
};
\addlegendentry{$M=1$}

\iffalse
\addplot [color=m4pq, mark=*, mark options={solid,m4pq}]
table[row sep=crcr]{%
	36	3.0017\\
	68	2.7291\\
	102	2.3122\\
	138	1.9902\\
	176	1.9802\\
	216	1.9539\\
	258	1.9442\\
};
\addlegendentry{$M=4; Q=P$}
\fi

\addplot [color=nn2, mark=diamond*, mark options={solid,nn2}]
table[row sep=crcr]{%
	5 	2.9797\\	% 1 
	12	2.9095\\	%2
	21	2.9732\\	%3
	32	2.2891\\	%4
	45	2.1186\\
	60	 1.9872\\
	77	1.9063\\
	96 	1.9166\\
};
\addlegendentry{$K=2$}

\addplot [color=nn, mark=diamond*, mark options={solid, nn}]
table[row sep=crcr]{%
	4	2.9454\\ 	% 1 neuron
	8	2.7549\\ 	% 2 neurons
	12	2.7357\\ 	% 3 neurons
	16	2.5547\\ 	% 4
	20	2.3683\\ 	 	% 5
	24	2.3100\\	% 6
	28	2.3016\\	% 7
	32	2.2198\\	% 8
	36	2.2257\\	% 9
	40	2.1936\\	% 10
	44	2.1394\\	% 11
	48	2.1122\\	% 12
	52	2.1786\\	% 13
	56	1.99\\
	60	2.1192\\
	64	2.0156\\
	68	 1.9999\\
	72	2.0352\\
	76	1.9212\\
	80	1.9245\\
	100	1.8835\\%	120	 1.9237\\
	124	1.9584\\
	240	 1.8988\\
};
\addlegendentry{$K=1$}

\addplot [mark=star, mark size=5pt, mark options={solid,nn}, forget plot]
coordinates {(24,	2.310)};
\addplot [mark=star, mark size=5pt, mark options={solid,m1}, forget plot]
coordinates {(27, 	2.0256)};

\addplot [mark=star, mark size=5pt, mark options={solid,nn}, forget plot]
coordinates {(56, 1.99)};
\addplot [mark=star, mark size=5pt, mark options={solid,m2}, forget plot]
coordinates {(66, 1.95947582666292)};


\end{axis}
\end{tikzpicture}%
	\caption{Example spectrum for the $M=4$ polynomial and $K=1$ \ac{NN}. Each of these use around 80 multiplications per time-domain input sample to the \ac{DPD}.}
	\label{fig:psd}
\end{figure}

\blindtext[5]

\section{LaTeX Hints}
\label{sec:hints}

\Cref{L1,L2} show listings typeset using the \texttt{lstlisting} environment.

\begin{lstlisting}[
  % one can adjust spacing here if required
  % aboveskip=2.5\baselineskip,
  % belowskip=-.8\baselineskip,
  caption={Example Java Listing},
  label=L1,
  language=Java,
  float]
public class Hello {
    public static void main (String[] args) {
        System.out.println("Hello World!");
    }
}
\end{lstlisting}

\begin{lstlisting}[
  % one can adjust spacing here if required
  % aboveskip=2.5\baselineskip,
  % belowskip=-.8\baselineskip,
  caption={Example XML Listing},
  label=L2,
  language=XML,
  float]
<example attr="demo">
  text content
</example>
\end{lstlisting}

\begin{figure}
  \includegraphics[width=.5\textwidth]{example-grid-100x100bp}
  \caption{Simple Figure. \cite[based on][]{memory_polynomial}}
  \label{fig:simple}
\end{figure}

\begin{figure*}
  \centering
  \includegraphics[width=.6\textwidth]{example-image-16x9}
  \caption{16x9 Figure}
  \label{fig:16x9}
\end{figure*}

\begin{figure*}[!b]
  \centering
  \subfloat[Case I]{\includegraphics[width=.4\textwidth]{example-image-a}%
    \label{fig_first_case}}
  \hfil
  \subfloat[Case II]{\includegraphics[width=.4\textwidth]{example-image-b}%
    \label{fig_second_case}}
  \caption{Simulation results for the network.}
  \label{fig_sim}
  % Note that often IEEE papers with subfigures do not employ subfigure
  % captions (using the optional argument to \subfloat[]), but instead will
  % reference/describe all of them (a), (b), etc., within the main caption.
  % Be aware that for subfig.sty to generate the (a), (b), etc., subfigure
  % labels, the optional argument to \subfloat must be present. If a
  % subcaption is not desired, just leave its contents blank,
  % e.g., \subfloat[].
\end{figure*}

\begin{table}
  \caption{Simple Table}
  \label{tab:simple}
  \centering
  \begin{tabular}{@{}ll@{}}
    \toprule
    Heading1 & Heading2 \\
    \midrule
    One      & Two      \\
    Thee     & Four     \\
    \bottomrule
  \end{tabular}
\end{table}

\begin{algorithm}
	\begin{algorithmic}[1]
		\State $i \gets \text{Block Index}$  	
		\State $n \gets \text{Thread Index}$
		\State \textbf{load} $\boldsymbol \beta^{(i)}$ from constant memory
		\State $m = 0; p = 1;$ \Comment{Memory and polynomial indexes}
		\For{$m<M$} \Comment{Loop over memory depth}
		\State $y \gets x^{(i)}(n-m)$ \Comment{Loaded from global memory}
		\State $\mathbf{z}[m] = y$ 
		\State $a = |y|^2$
		\State $b = y \cdot a$
		\For{$p<=P$}
		\State $\mathbf{z}[pM+m] = b$ 
		\State $b = b \cdot a$
		\EndFor 
		\EndFor
		\State $\hat{x}^{(i)}(n) = \mathbf{z} \cdot \boldsymbol \beta^{(i)}$ 	
		\State \textbf{return} $\hat{x}^{(i)}(n)$
	\end{algorithmic} 
	\caption{Example Algorithm }
	\label{ker:1}
\end{algorithm}

cref Demonstration: Cref at beginning of sentence, cref in all other cases.

\Cref{fig:simple} shows a simple fact, although \cref{fig:simple} could also show something else.
\Cref{fig:16x9} shows an 16x9 image spanning two columns.
\Cref{fig_first_case} is the first subfloat, whereas \Cref{fig_second_case} is the second one.

\Cref{tab:simple} shows a simple fact, although \cref{tab:simple} could also show something else.

\Cref{sec:intro} shows a simple fact, although \cref{sec:intro} could also show something else.

Brackets work as designed:
<test>
One can also input backquotes in verbatim text: \verb|`test`|.

The symbol for powerset is now correct: $\powerset$ and not a Weierstrass p ($\wp$).

\begin{inparaenum}
  \item All these items...
  \item ...appear in one line
  \item This is enabled by the paralist package.
\end{inparaenum}

``something in quotes'' using plain tex or use \enquote{the enquote command}.

You can now write words containing hyphens which are hyphenated (application"=specific) at other places.
This is enabled by an additional configuration of the babel package.
In case you write \enquote{application-specific}, then the word will only be hyphenated at the dash.

 \begin{figure}[t]
 	\centering
 	\begin{tikzpicture}[shorten >=1pt,->,draw=black!50, node distance=\layersep]
 	\tikzstyle{every pin edge}=[<-,shorten <=1pt]
 	\tikzstyle{neuron}=[circle,fill=black!25,minimum size=17pt,inner sep=0pt]
 	\tikzstyle{input neuron}=[neuron, fill=green!50];
 	\tikzstyle{output neuron}=[neuron, fill=red!50];
 	\tikzstyle{hidden neuron}=[neuron, fill=blue!50];
 	\tikzstyle{annot} = [text width=4em, text centered]
 	
 	% Draw the input layer nodes
 	\node[input neuron, pin=left:$\Re(x)$] (I-1) at (0, -0.5) {};
 	\node[input neuron, pin=left:$\Im(x)$] (I-2) at (0,-1.5) {};
 	
 	% Draw the hidden layer nodes
 	\foreach \name / \y in {1,...,2}
 	\path[yshift=0.5cm]
 	node[hidden neuron] (H-\name) at (\layersep,-\y cm) {};
 	
 	\node[align=left, rotate=90]  at (\layersep,-1 cm) {...};
 	\node[align=left]  at (\layersep+0.2cm,-1 cm) {$N$};
 	
 	
 	% Draw the output layer node
 	\node[output neuron,pin={[pin edge={->}]right:$\Re(y)$}, right of=H-1] (O-1) {};
 	\node[output neuron,pin={[pin edge={->}]right:$\Im(y)$}, right of=H-2] (O-2) {};
 	
 	% Connect every node in the input layer with every node in the
 	% hidden layer.
 	\foreach \source in {1,...,2}
 	\foreach \dest in {1,...,2}
 	\path (I-\source) edge (H-\dest);
 	
 	% Connect every node in the hidden layer with the output layer
 	\foreach \source in {1,...,2}
 	\path (H-\source) edge (O-1);
 	
 	\foreach \source in {1,...,2}
 	\path (H-\source) edge (O-2);
 	
 	
 	% Linear bypass
 	\path[bend left,->] (I-1) edge (O-1);
 	\path[bend right,->] (I-2) edge (O-2);
 	
 	% Annotate the layers
 	\node[annot,above of=H-1, node distance=1cm] (hl) {Hidden layers};
 	\node[annot,left of=hl] {Input layer};
 	\node[annot,right of=hl] {Output layer};
 	\end{tikzpicture}
 	\caption{General structure of the \ac{DPD} and \ac{PA} neural networks. There are two input and output neurons for the real and imaginary parts of the signal, $N$ neurons per hidden layer, and $K$ hidden layers. The inputs are directly added to the output neurons so that the hidden layers concentrate on the nonlinear portion of the signal.}
 	\label{fig:NN_arch}
 \end{figure}

\section{Lorem ipsum}
\label{sec:loremipsum}
\lipsum[1-4]

\section{Conclusion and Outlook}
\label{sec:outlook}

\blindtext

% use section* for acknowledgment
\ifCLASSOPTIONcompsoc
  % The Computer Society usually uses the plural form
  \section*{Acknowledgments}
\else
  % regular IEEE prefers the singular form
  \section*{Acknowledgment}
\fi

This work received support from the \href{https://latextemplates.github.io/IEEE/}{IEEE LaTeX Template}.


In the bibliography, use \texttt{\textbackslash textsuperscript} for ``st'', ``nd'', \ldots:
E.g., \enquote{The 2\textsuperscript{nd} conference on examples}.
When you use \href{https://www.jabref.org}{JabRef}, you can use the clean up command to achieve that.
See \url{https://help.jabref.org/en/CleanupEntries} for an overview of the cleanup functionality.

% trigger a \newpage just before the given reference
% number - used to balance the columns on the last page
% adjust value as needed - may need to be readjusted if
% the document is modified later
%\IEEEtriggeratref{8}
% The "triggered" command can be changed if desired:
%\IEEEtriggercmd{\enlargethispage{-5in}}

% Enable to reduce spacing between bibitems (source: https://tex.stackexchange.com/a/25774)
% \def\IEEEbibitemsep{0pt plus .5pt}

\bibliographystyle{IEEEtranN} % IEEEtranN is the natbib compatible bst file
% argument is your BibTeX string definitions and bibliography database(s)
\bibliography{refs}


\end{document}
